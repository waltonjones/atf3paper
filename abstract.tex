The \emph{Drosophila} olfactory system is highly stereotyped in form and function; olfactory sensory neurons (OSNs) expressing a specific odorant receptor (OR) always appear in the same antennal location and the axons of OSNs expressing the same OR converge on the same antennal lobe glomeruli. Although some transcription factors have been implicated in a combinatorial code specifying OR expression and OSN identity, it is clear other players remain unidentified. To mitigate some of the challenges of genome-wide screening, we propose a two-tiered approach comprising a primary ``pooling'' screen for miRNAs whose tissue-specific over-expression causes a phenotype of interest followed by a focused secondary screen using gene-specific RNAi. Since miRNAs down-regulate their target mRNAs, miRNA over-expression phenotypes should be attributable to target loss-of-function. Since miRNA-target pairing is sequence-dependent, predicted targets of miRNAs identified in the primary screen are candidates for the secondary screen. Since miRNAs are short, however, miRNA misexpression will likely uncover non-biological miRNA-target relationships. Rather than focusing on miRNA function itself where these non-biological relationships could be misleading, we propose using miRNAs as tools to focus a more traditional RNAi-based screen. Here we describe a proof-of-concept miRNA-based screen that uncovers a role for Atf3 in the expression of the odorant receptor Or47b.