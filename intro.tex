\section*{Introduction}

The transgenic RNAi fly stock libraries (e.g., the \href{http://stockcenter.vdrc.at/control/rnailibrary}{Vienna Drosophila RNAi library} \cite{Oppel_Scheiblauer_et_al__2007} and the \href{http://www.flyrnai.org/TRiP-HOME.html}{Transgenic RNAi Project (TRiP)}) have been a tremendous boon to the \emph{Drosophila} community because they permit tissue-specific knockdown of almost all genes in the genome.
These resources permit genome-wide screens for genes associated with almost any phenotype of interest.
Unfortunately, the sheer size of these libraries---more than 22,000 stocks in the case of the Vienna library---means performing such screens remains labor-intensive and tedious.
In this paper, we describe our development of a two-tiered screening protocol comprising an initial pooling screen using miRNA over-expression that generates a list of candidate genes involved in a phenotype of interest and a secondary screen using gene-specific RNAi that narrows this list of candidates to the responsible target gene(s).
We suggest that this protocol can sometimes accelerate the identification of novel genes involved in a broad range of phenotypes.

MicroRNAs are short, endogenous, single-stranded RNA molecules that act in the context of the miRISC protein complex to either inhibit translation or induce the degradation of target mRNAs \cite{BARTEL_2004}.
Since the miRNA-target mRNA relationship is determined primarily by a short seed sequence at the 5' end of each miRNA \cite{ones-Rhoades_Bartel_Burge_2003,Lai_2002}, the complement of which may occur in multiple copies scattered over the genome, many miRNAs are capable of down-regulating multiple targets.
The relationship between a miRNA seed sequence and its complements in the open reading frames and 3'-untranslated regions (3'-UTRs) of target mRNAs spurred the development of bioinformatic algorithms that convert mature miRNA sequences into lists of potential mRNA targets \cite{Rajewsky_2006}.
These lists of candidate targets, however, are plagued by large numbers of false positives because the algorithms that generate them can fully account for neither the precise spatial and temporal patterns of miRNA and target mRNA expression nor target site availability.
In other words, a miRNA may be capable of down-regulating a particular target and never actually do so, either because the two are never simultaneously expressed in the same tissue or because RNA-binding proteins or RNA folding render the target site inaccessible.
It also follows that miRNA over-expression in arbitrary tissues using the binary GAL4/UAS expression system would likely lead to non-biological miRNA-target mRNA pairings.
Rather than seeing these pairings as a potential drawback of using a library of UAS-miRNA stocks, we expect they can be useful as part of a two-tiered screening system.

We previously generated a library of 131 UAS-miRNA fly stocks that permit tissue-specific over-expression of 144 \emph{Drosophila} miRNAs \cite{suh_2015aa}.
In this study, we sought to use these UAS-miRNA stocks to validate the concept of a two-tiered miRNA-based screen in the \emph{Drosophila} olfactory system.

The olfactory receptor neurons (ORNs) of adult \emph{Drosophila} are housed in porous hair-like sensilla that cover the paired antennae and maxillary palps.
Olfactory sensilla are divided into 3 main classes by their shape and 17 subclasses by their odor response profiles \cite{Couto_Alenius_Dickson_2005}.
The odor response profile of an ORN is determined by its expression of the obligatory olfactory co-receptor Orco and one or very few of the adult odor-specific odorant receptors (ORs) \cite{Vosshall_Stocker_2007}.
The spatial arrangement of the 17 subclasses of adult olfactory sensilla on the antenna, the arrangement of the ORNs themselves, the precise pattern of OR expression, and the wiring of the antennal OSNs into the appropriate glomeruli of the antennal lobe are all highly stereotyped from fly to fly, indicating well-orchestrated developmental control of every step in the process.

Recently Jafari et al. reported the results of a large-scale RNAi screen that identified seven transcription factors, permutations of which determine the odorant receptor expressed by each population of olfactory neurons in the adult \emph{Drosophila} antenna.
Despite the success of their screen, Jafari et al. extrapolated from the complexity of the fly olfactory system and estimated that at least three more unidentified transcription factors are likely part of the combinatorial code that determines OR expression \cite{Brochtrup_Hummel_Alenius_2012}.

In their screen, Jafari et al. combined the Peb-GAL4 driver line, which is strongly expressed in peripheral sensory neurons including the antennal olfactory neurons, with a pair of OR promoter fusions (i.e., Or47b and Or92a) to a membrane-tethered GFP that act as reporters of OR expression.
We obtained these lines and by crossing them to our library of UAS-miRNA stocks we were able to identify miRNAs whose over-expression eliminates Or47b expression, Or92a expression, or both.
We chose to proceed with the miRNAs that affect Or47b expression (i.e., bantam, miR-2a-2, miR-33, miR-263a, miR-308, miR-973/974, and miR-2491).
We then used existing bioinformatic tools to generate lists of their putative mRNA targets, compare the lists for overlap, and define a short list of candidate genes for a small follow-up RNAi screen.
In this follow-up screen, we identified a previously unknown role for Activating transcription factor 3 (Atf3) in the expression of Or47b.
