\subsection*{Secondary RNAi screen for miRNA targets that modify OR expression}

Following the screening protocol outlined in \textbf{Fig. \ref{fig:1}A}, we next compiled lists of possible target genes for each of the miRNAs whose over-expression eliminates Or47b expression.
We initially compared the outputs of multiple target prediction algorithms (i.e., TargetScan, miRanda, and PITA), as this is standard practice \cite{am_Namkoong_Lee_Chung_Kim_2009,agen_Okamura_Perrimon_Lai_2007}.
\textbf{Figure \ref{fig:2}A} compares the numbers of candidate targets predicted for each miRNA that induces loss of Or47b by the \href{http://www.targetscan.org/fly_12/}{TargetScan} \cite{ohnston_Kellis_Bartel_Lai_2007}, \href{http://www.microrna.org}{miRanda} \cite{Gaul_Tuschl_Sander_Marks_2003}, and \href{http://genie.weizmann.ac.il/pubs/mir07/mir07_dyn_data.html}{PITA} \cite{ino_Unnerstall_Gaul_Segal_2007} algorithms.

Typically, miRNA-based screens using GAL4 lines that drive expression in well-characterized tissues would allow a list of predicted miRNA targets to be prioritized based on expression in the tissue of interest.
In this case, however, without access to transcriptome data for the olfactory neurons of interest, we decided to prioritize our secondary RNAi screen by looking for overlap between the lists of candidate targets for each miRNA whose over-expression eliminated Or47b expression.
We decided to focus on TargetScan because it predicts miRNA seed matches in both 3'-UTRs and open reading frames as well as provides a way to query more recently identified miRNAs like miR-2491.
PITA does not provide easy access to this option, and the miRanda algorithm produces too many predicted targets.
TargetScan predicts a total of 1399 unique targets for the seven UAS-miRNA lines that eliminate Or47b expression.
While 231 of these unique candidates are shared between the prediction lists for two of the miRNAs, only 22 targets appear on the lists for three of the miRNAs (\textbf{Fig. \ref{fig:2}B-C}).

We next screened through this list of candidate targets appearing on the prediction lists for multiple miRNAs by combining gene-specific UAS-candidate-IR (inverted repeat) lines with the control Peb-GAL4; Or47b\textgreater{}CD8::GFP, Or92a\textgreater{}CD8::GFP genotype.
In this secondary screen, knockdown of responsible targets can be expected to mimic the miRNA-induced loss of Or47b.
This is how we were able to identify a role for Activating Transcription Factor 3 (Atf3 or A3-3) in the expression of Or47b.
Atf3 appears on the predicted target lists for three miRNAs whose over-expression eliminates Or47b expression (i.e., miR-2a-2, miR-33, and miR-2491).
\textbf{Figure \ref{fig:2}D} indicates the positions of the miR-2a-2 and miR-33 binding sites in the Atf3 3'-UTR.
The miR-2491 binding site in the 3'-UTR of Atf3 is conserved across several species of \emph{Drosophila}, but because it is not part of the standard TargetScan search parameters it does not appear in the exportable graphic used to make \textbf{Figure \ref{fig:3}D}.
Note also that over-expression of several miRNAs (e.g., let-7, miR-285, etc.) with predicted binding sites in the Atf3 3'-UTR fails to eliminate Or47b expression, presumably due to issues with binding site accessibility.
This fact that it is currently impossible to predict which of the clear miRNA seed match sites represent true positives is what makes a follow-up gene-specific RNAi screen necessary.
\textbf{Figure \ref{fig:2}E} shows the result of our secondary RNAi screen in which knockdown of Atf3 dramatically and specifically reduces the expression of the Or47b reporter (blue arrowhead) and not the Or92a reporter (red arrowhead).
Furthermore, loss of Atf3 eliminates Or47b expression as assessed via mixed probe \emph{in situ} hybridization while leaving Or92a expression unaffected (compare \textbf{Fig. \ref{fig:2}F} to \textbf{Fig. \ref{fig:1}C}).
Since we are using miRNAs as a screening tool, the precise mechanisms by which they produce their phenotypes are irrelevant to the results of the follow-up screen.
Still, in the interest of completeness, we confirmed that the over-expression of miR-2a-2 reduces the level of Atf3 in antennal cDNAs as detected by qPCR (\textbf{Fig. \ref{fig:2}G}).
We were also able to confirm that the loss of Or47b expression induced by miR-2a-2 over-expression is rescued by introduction of an Atf3::eGFP fusion that lacks its 3'-UTR and thus the miR-2a-2 binding site (\textbf{Fig. \ref{fig:2}H}).

It is interesting to note that Eip93f, the only other transcription factor whose loss-of-function eliminates Or47b expression, appears on the candidate target lists of two miRNAs whose over-expression eliminates expression of the Or47b reporter (i.e., miR-308 and miR-2491).
This means we could have identified it in a second round of gene-specific RNAi screening had the list of candidate targets appearing on three lists failed to identify a true positive.
