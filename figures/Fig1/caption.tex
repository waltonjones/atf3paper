{\bf Primary screen for miRNAs whose over-expression alters OR reporter expression in the antennal lobe.}

\textbf{A}.
A two-tiered, miRNA-based screening workflow can accelerate loss-of-function genetic screens by limiting the number of necessary gene-specific RNAi crosses.
\textbf{B}.
Antennal lobe (AL) staining of the control Peb-GAL4/+; Or47b\textgreater{}CD8::GFP, Or92a\textgreater{}CD8::GFP/+ controls shows strong GFP staining in both the Or47b glomerulus VA1lm (blue arrowhead) and the Or92a glomerulus VA2 (red arrowhead).
\textbf{C}.
AL staining for 3 of the 7 miRNAs whose over-expression induces loss of the Or47b reporter (blue arrowheads).
\textbf{D}.
AL staining for 3 of the 4 miRNAs whose over-expression induces loss of the Or92a reporter (red arrowheads).
\textbf{E}.
AL staining for the 3 miRNAs whose over-expression induces loss of both the Or47b and Or92a reporters (blue and red arrowheads).
\textbf{F}.
Mixed probe \emph{in situ} on the same control genotype as in B.
Or92a-expressing cells are circled in red, while Or47b-expressing cells are circled in blue.
\textbf{G}.
Mixed Or47b/Or92a probe \emph{in situs} on antennae of the genotypes shown in D confirm the loss of Or47b expression induced by miR-263a, miR-2a-2, and miR-2491.
\label{fig:1}