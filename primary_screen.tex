\subsection*{Primary screen for miRNAs that modify OR expression}

As stated, we settled on \emph{Drosophila} olfactory neurons to demonstrate our miRNA-based screening strategy.
We chose Pbl-GAL4 to drive expression of UAS-miRNAs in adult olfactory neurons because it is expressed strongly in peripheral sensory neurons beginning 12-18 hours after pupation \cite{dnik_Dickson_Luo_Komiyama_2007}.
By combining Pbl-GAL4 with fusions of two odorant receptor promoters (i.e., Or92a and Or47b) to the coding sequence for a membrane-tethered GFP and crossing in the UAS-miRNAs, we were able to visualize the effects of miRNAs on OR expression.
Or92a is expressed in large basiconic sensilla in a class of OSN called ab1B, which respond to 2,3-butanedione \cite{de_Bruyne_Foster_Carlson_2001}.
Or47b is expressed in trichoid sensilla in a class of OSNs called at4, which respond to socially relevant fly-derived odors \cite{vanderGoesvanNaters:2007cq}.
We chose Or92a and Or47b because they are expressed in olfactory neurons that innervate prominent, well-separated glomeruli in the antennal lobe (i.e., VA2 and VA1lm) and whose cell bodies are easily distinguishable upon antennal RNA \emph{in situ} hybridization.

For the primary miRNA over-expression screen, we crossed a stable homozygous stock containing Pbl-GAL4 and the Or47b and Or92a promoter fusions to all of our UAS-miRNA lines.
We observed that Pbl-GAL4-mediated over-expression of 20 of 131 miRNA lines is lethal or nearly lethal (see Supplemental Table 1).
For the remaining miRNAs, we compared the levels of the GFP reporters of OR expression in the antennal lobes of flies over-expressing miRNAs with a heterozygous control (\textbf{Fig. \ref{fig:1}B-C}).
In doing so, we were able to divide miRNAs into three categories: those that reduce the intensity of the Or47b reporter in VA1lm (\textbf{Fig. \ref{fig:1}D}, blue arrowheads), the Or92a reporter in VA2 (\textbf{Fig. \ref{fig:1}E}, red arrowheads), or both (\textbf{Fig. \ref{fig:1}F}).

We decided to focus our follow-up on the miRNAs that reduce the expression of the Or47b reporter because only one transcription factor, Eip93f, is associated with Or47b expression \cite{Brochtrup_Hummel_Alenius_2012}.
In addition, some of the miRNAs that cause loss of Or92a also cause a less significant loss of Or47b, meaning the loss of Or47b phenotypes are less ambiguous.
Before proceeding to miRNA target identification, we verified the miRNA-induced loss of Or47b using RNA \emph{in situ} hybridization with mixed riboprobes against both Or47b and Or92a.
All together we identified seven miRNA lines (Pictured: miR-263a, miR-2a-2, and miR-2491; Not shown: bantam, miR-33, miR-308, and miR-973/974) that reduce the expression of the Or47b reporter.
We were able to verify that these lines eliminate Or47b (circled in blue) expression when combined with the Peb-GAL4 driver while leaving Or92a (circled in red) expression intact (\textbf{Fig. \ref{fig:1}G}).
