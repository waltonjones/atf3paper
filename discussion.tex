\section*{Discussion}

Here we report a simple proof-of-concept for a two-tiered, miRNA-based screening system designed to accelerate the process of genetic screening in \emph{Drosophila}.
We generated a collection of transgenic fly stocks that permit the tissue-specific over-expression of miRNAs.
Since miRNAs down-regulate their targets, we reasoned that miRNAs could be used for a pooling pre-screen that points the way to a much smaller list of RNAi stocks for follow-up screening than would normally be necessary.
We demonstrated this screening method in the olfactory system where we identified a novel role for the transcription factor Atf3 in the expression of the socially relevant odorant receptor Or47b.

In addition, Eip93f, the only other transcription factor known to be involved in Or47b expression, is also a predicted target of some of the miRNAs we found to cause loss of Or47b expression.
In other words, we were able to validate our screening system on a known transcription factor and rapidly (i.e., in a small number of genetic crosses) identify a previously unknown player in OR expression, Atf3.

The clear advantage to a two-tiered miRNA-based screening strategy is a reduction in the total number of crosses necessary to identify a gene associated with a specific phenotype.
Genetic screening always depends on a limited human resource---the patience and endurance of the researcher.
In traditional genetic screens the assay for the screened phenotype must be as simple and streamlined as possible to minimize researcher fatigue.
In our proof-of-principle screen in the olfactory system, we were able to identify Atf3 in roughly 160 genetic crosses.
With this level of acceleration, it becomes feasible to design screens with more complicated or time-consuming assays.

Despite providing ample evidence of a role for Atf3 in Or47b expression, it remains unclear whether Atf3 is acting directly on the Or47b promoter.
Miller and Carlson generated a series of GAL4 lines from truncated Or47b promoters.
They observed that promoters of 7.6 kilobases down to 419 base pairs drive expression in a small population of neurons in the distolateral antenna similar to the RNA \emph{in situ} pattern circled in blue in \textbf{Fig. \ref{fig:1}F}.
Further truncation expands the range of labeled cells into the proximal antenna and the maxillary palps \cite{Miller_Carlson_2010}.
Their results suggest the existence of an essential repressor binding site in the Or47b promoter between -419 and -342 bp from the transcription start site (TSS) and an essential activator binding site between -219 bp and -119 bp from the TSS.
Unfortunately, there is no clear binding site in the Or47b promoter that matches the only published \cite{a_Bohmann_Jindra_Uhlirova_2009} consensus sequence recognized by \emph{Drosophila} Atf3, TGACGTCA.
There is a published \cite{Brodesser_Jindra_Uhlirova_2012} consensus site that would be recognized by mammalian Atf3 (TGAYRTCA) 918 bp upstream of the TSS, but this falls outside of the -219 to -119 bp range suggested by Miller and Carlson \cite{Miller_Carlson_2010}.
Thus, only further experiments will determine the exact mechanism of action by which Atf3 regulates Or47b expression.

Although our proof-of-concept screen was successful, it has not escaped our attention that miRNA-based screening is inherently biased and will not be suitable for screening in every tissue or for every phenotype.
One type of bias stems from the strict use of endogenous \emph{Drosophila} miRNAs rather than designing short synthetic miRNAs that can target a wider range of genes.
Even with a collection of synthetic miRNAs, though, any form of miRNA-based screening will be biased toward identifying genes with longer open reading frames and longer 3'-UTRs.
Another potential problem with a miRNA-based approach is the case where a single miRNA produces a synthetic phenotype that is not attributable to a single or predominant target mRNA.
Still, we hope that the miRNA-based screening method we describe here will be another useful addition to the geneticist's toolbox.