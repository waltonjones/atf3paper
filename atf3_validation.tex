\subsection*{Validation of a role for Atf3 in Or47b expression}

Atf3 is a member of the basic leucine zipper (bZIP) family of transcription factors.
In \emph{Drosophila}, Atf3 functions in immune and metabolic homeostasis as well as abdominal morphogenesis \cite{Brodesser_Jindra_Uhlirova_2012,a_Bohmann_Jindra_Uhlirova_2009}.
Consistent with a role for Atf3 in Or47b expression rather than olfactory neurogenesis, knockdown of Atf3 does not affect OSN morphology, Orco expression, or Orco localization (\textbf{Fig. \ref{fig:3}A}).
We also confirmed that Atf3 is expressed in Or47b neurons by combining an Atf3 protein fusion to eGFP under the transcriptional control of the Atf3 promoter with an Or47b-GAL4 line driving the red fluorescent marker UAS-TdTomato (\textbf{Fig. \ref{fig:3}B}).

The fact that the other transcription factors implicated in OR expression (e.g., Acj6, Pdm3, Xbp1, Eip93f, etc.) work together as part of a combinatorial code that includes both transcriptional activators and repressors \cite{Brochtrup_Hummel_Alenius_2012}, suggests that broad antennal over-expression of Atf3 may expand the expression pattern of Or47b.
On the contrary, the combination of UAS-Atf3 with the olfactory co-receptor Orco-GAL4 line, which drives expression in most antennal OSNs, has no effect on the Or47b expression pattern (\textbf{Fig. \ref{fig:3}C}).
This suggests other OSN populations express sufficient repressive transcription factors to prevent Or47b expression or lack an Atf3 binding partner required for Or47b activation.
It is also possible that chromatin modifications or other unknown changes arise between OSN birth and final differentiation that somehow prevent the functional binding of ectopic Atf3 to the promoters of other ORs.

We used Peb-GAL4 in our two-tiered miRNA-based screen because it drives expression in peripheral sensory neurons including those of the developing antenna beginning 12-18 hours after puparium formation (APF) \cite{dnik_Dickson_Luo_Komiyama_2007}.
Since Peb-GAL4 expression begins soon after the birth of the OSNs and long before the earliest OR expression begins at 50 hours APF, it is possible that Atf3 plays a developmental role in Or47b-expressing neurons.
To rule out this possibility, we repeated the Atf3 loss-of-function experiment using Orco-GAL4.
Orco-GAL4 begins to drive expression in antennal OSNs at roughly 80 hours APF, after olfactory development is complete \cite{s_Chiappe_Amrein_Vosshall_2004}.
Like the result with Peb-GAL4, knockdown of Atf3 using Orco-GAL4 eliminates Or47b expression, and this loss of Or47b expression can be partially rescued by re-introduction of Atf3 (\textbf{Fig. \ref{fig:3}D}).

In further confirmation of Atf3's role in OR expression, we performed an RNA \emph{in situ} hybridization experiment using mixed riboprobes against both Or92a and Or47b on Atf3-null mutants.
Although the \emph{atf3\textsuperscript{76}} allele is a recessive lethal mutation \cite{Brodesser_Jindra_Uhlirova_2012}, we were able to obtain a few hemizygous male escapers.
While Or92a expression in these mutant antennae remains intact, the expression of Or47b is completely lost (\textbf{Fig. \ref{fig:3}E, left}).
The combination of the \emph{atf3\textsuperscript{76}} allele with the Atf3::eGFP BAC transgene, however, rescues both the lethality and the expression of Or47b (\textbf{Fig. \ref{fig:3}E, right}).

Finally, because we observed a population of OSNs labeled by Atf3::eGFP and not by Or47b-G4>TdTomato (\textbf{Fig. \ref{fig:3}B}), we asked whether Atf3 is involved in the expression of other ORs.
We knocked-down the expression of Atf3 in most OSNs using Orco-GAL4 (\emph{Orco-Gf>atf3-IR1}) and performed antennal RNA \emph{in situ} hybridizations using riboprobes specific to several more ORs known to be expressed in the adult antenna (i.e., Or13a, Or22a, Or22b, Or42b, Or43b, Or59b, Or85b, Or98a, and Orco).
Of these, only Or43b is lost upon knockdown of Atf3 like Or47b (\textbf{Fig. \ref{fig:3}F}).
